%% LyX 2.0.0 created this file.  For more info, see http://www.lyx.org/.
%% Do not edit unless you really know what you are doing.
\documentclass[12pt, english]{kuthesis}
\usepackage{amsmath, amssymb, amsfonts}
\usepackage{mathptmx}
\renewcommand{\sfdefault}{lmss}
\renewcommand{\ttdefault}{lmtt}
\usepackage[T1]{fontenc}
\usepackage[utf8]{inputenc}
\usepackage{listings, geometry}
\geometry{verbose,tmargin=1in,bmargin=1in,lmargin=1in,rmargin=1in}
\setcounter{secnumdepth}{3}
\setcounter{tocdepth}{3}
\usepackage{color}
\usepackage{graphicx}
\usepackage[doublespacing]{setspace}
\usepackage{esint}
\usepackage[authoryear]{natbib}
\usepackage{amsthm, verbatim, bm}
\usepackage{floatrow}
\usepackage{listings}
\usepackage{mathtools}
\usepackage{calc}
\usepackage{enumitem}
\usepackage{slashbox}

\newcommand{\ds}{\displaystyle}

\newfloatcommand{capbtabbox}{table}[][\FBwidth]


\newcommand{\dfn}[1]{\textit{#1}}
\newcommand{\dc}{,\dotsc,}
\DeclareMathOperator{\vrt}{vert}
\makeatletter

%%%%%%%%%%%%%%%%%%%%%%%%%%%%%% Allow overlay of characters
\fboxsep=1pt
\newcommand{\olay}[2]{\makebox[\widthof{$#2$}][c]{$#1$}}
\newcommand{\ze}{\olay{0}{+}}


%%%%%%%%%%%%%%%%%%%%%%%%%%%%%% LyX specific LaTeX commands.
\providecommand{\LyX}{L\kern-.1667em\lower.25em\hbox{Y}\kern-.125emX\@}
%% Because html converters don't know tabularnewline
\providecommand{\tabularnewline}{\\}
%% A simple dot to overcome graphicx limitations
\newcommand{\lyxdot}{.}


%%%%%%%%%%%%%%%%%%%%%%%%%%%%%% User specified LaTeX commands.

\newcommand{\tri}{\rotatebox[origin=c]{180}{\(\nabla\)}}
\newcommand{\mb}[1]{\mathbb{#1}}
\newcommand{\mc}[1]{\mathcal{#1}}
\newcommand{\mbf}[1]{\mathbf{#1}}
\newcommand{\scr}[1]{\mathcal{#1}}
\newcommand{\mf}[1]{\mathfrak{#1}}
\newcommand{\fl}[1]{\scr F(#1)}
\newcommand{\ph}{\ensuremath{\phantom{.}}}
\DeclareMathAlphabet{\mathpzc}{OT1}{pzc}{m}{it}

\newcommand{\sbset}{\subseteq}

\newcommand{\mt}{\varnothing}
\newcommand{\la}{\lambda}
\newcommand{\RR}{\mb R}
\newcommand{\Tr}[1]{{#1}^{\mathrm{T}}}
\newcommand{\R}[1]{\RR^{#1}}
\newcommand{\Sp}[1]{\mb S^{#1}}
\newcommand{\ol}[1]{\overline{#1}}
\newcommand{\wt}[1]{\widetilde{#1}}
\newcommand{\ve}[1]{\bm{#1}}
\newcommand{\brac}[1]{\left[#1\right]}
\newcommand{\floor}[1]{\left\lfloor#1\right\rfloor}
\newcommand{\ceil}[1]{\left\lceil#1\right\rceil}
\newcommand{\ip}[2]{\left\langle#1,#2\right\rangle}
\newcommand{\abs}[1]{\left\lvert#1\right\rvert}
\newcommand{\norm}[1]{\left\lVert#1\right\rVert}
\newcommand{\card}[1]{\abs{#1}}
\newcommand{\tast}{\ensuremath{\ast}}
\newcommand{\join}{\vee}
\newcommand{\N}{\mb N}
\newcommand{\B}[2]{B_{#1}\left(#2\right)}
\newcommand{\eps}{\varepsilon}
\newcommand{\usv}[5]{\Tr{\begin{bmatrix}#1 & #2 & #3 & #4 & #5\end{bmatrix}}}
\newcommand{\sv}[5]{\pm \usv#1#2#3#4#5}
\newcommand{\lb}{\right.\\ &\phantom{= \left\{.\vphantom.\right.}\left.}
\newcommand{\zmat}[2]{[0]_{\substack{#1\\ #2}} }
\newcommand{\snl}{\right. \\ &\phantom{=\Biggl\{}\left.}
\newcommand{\nv}{^{-1}}

\newcommand{\cyc}[2]{C_{#2}(#1)}                            % cyclic polytope with #1 vertices in #2 dimensions
\newcommand{\simp}[1]{\Delta_{#1}}                          % #1-simplex
\newcommand{\cube}[1]{Q_{#1}}                               % #1-cube
\newcommand{\xp}[1]{X_{#1}}                                 % #1-crosspolytope
\newcommand{\rest}[2]{\left.#1\right\vert_{#2}}             % Restriction of G to W
\newcommand{\gr}[1]{\mc G(#1)}                              % The graph of a polytope #1
\newcommand{\conn}[1]{\kappa\left(#1\right)}                % connectivity of graph #1
\newcommand{\had}[1]{h\left(#1\right)}                      % Hadwiger numberof graph #1


\newcommand{\seta}[1]{\left\{#1\right\}}
\newcommand{\setb}[2]{\seta{#1\vphantom{#1 #2}\phantom{.}\right\vert\left.\vphantom{#1 #2}#2}}

\DeclareMathOperator{\spn}{span}
\DeclareMathOperator{\conv}{conv}
\DeclareMathOperator{\rref}{rref}
\DeclareMathOperator{\relint}{relint}
\DeclareMathOperator{\dep}{dep}
\DeclareMathOperator{\aff}{aff}
\DeclareMathOperator{\Ne}{N}
\DeclareMathOperator{\Po}{P}
\DeclareMathOperator{\sign}{sign}
\DeclareMathOperator{\supp}{supp}
\DeclareMathOperator{\pyr}{pyr}
\DeclareMathOperator{\prism}{prism}
\DeclareMathOperator{\galed}{gale}
\DeclareMathOperator{\sgn}{{\textsc{sign}}}
\DeclareMathOperator{\sepn}{sep}
\DeclareMathOperator{\binlog}{lb}

\newcommand{\Neg}[1]{\Ne(#1)}
\newcommand{\Pos}[1]{\Po(#1)}
\newcommand{\sep}[2]{\sepn\left(#1,#2\right)}

%%%%%%%%%%%  Horizontal shifter
\makeatletter
\newcommand*{\shifttext}[2]{%
  \settowidth{\@tempdima}{#2}%
  \makebox[\@tempdima]{\hspace*{#1}#2}%
}
\makeatother

%%%%%%%%%% amsthm Theorem styles

% Define `Theorem', and reset its counter after each chapter increment.
\newtheorem{Theorem}{Theorem}
\makeatletter
\@addtoreset{Theorem}{thechapter}
\makeatother

% Define `Conjecture', it has no counter.
\newtheorem*{Conjecture}{Conjecture}


% Define `Corollary', and reset its counter after each theorem increment.
\newtheorem{Corollary}{Corollary}
\makeatletter
\@addtoreset{Corollary}{Theorem}
\makeatother
% Define UBT, no counter
    \newtheorem*{UBT}{Upper Bound Theorem}
% Define GEC, no counter
    \newtheorem*{GEC}{Gale's Evenness Condition}
% Define Cor, no counter
    \newtheorem*{Cor}{Corollary}

% Define `Lemma', count on the same counter as Theorem.
    \newtheorem{Lemma}[Theorem]{Lemma}

\theoremstyle{definition}
% Define `Definition' and `Definftions', no counter.
    \newtheorem*{Definition}{Definition}
    \newtheorem*{Definitions}{Definitions}
% Define `Note', no counter.
    \newtheorem*{Note}{Note}
% Define `Remark', no counter.
    \newtheorem*{Remark}{Remark}
    \newtheorem*{Question}{Question}
% Define `OpenProblem', no counter.
    \newtheorem*{OpenProblem}{Open Problem}
% Define `Example', count on the same counter as Proposition.
    \newtheorem{Example}[Theorem]{Example}
% Define `Examples', count on the same counter as Proposition.
    \newtheorem{Examples}[Theorem]{Examples}

%   Proof Sketch environment
    \newenvironment{SketchProof}{\trivlist\item[]{\sc{Sketch of Proof}}.}%
    {\unskip\nobreak\hskip 1em plus 1fil\nobreak\(\Box\)
    \parfillskip=0pt%
    \endtrivlist}





%used to align decimals in tables according to APA style
\usepackage{dcolumn}
\usepackage{booktabs}

% Set the title and author info
\title{Graphs of Polytopes}
\author{William Espenschied}


\dept{Department of Mathematics}
\degreetitle{Doctor of Philosophy}
\papertype{Dissertation} %capitalization is important here
\committee{Margaret Bayer}{Jeremy Martin}{Jack Porter}{Saul Stahl}{Nancy Kinnersley}


\begin{comment}
\@ifundefined{showcaptionsetup}{}{%
 \PassOptionsToPackage{caption=false}{subfig}}
\usepackage{subfig}
\makeatother
\end{comment}

\datedefended{09-02-2014}
\dateapproved{09-02-2014}


\usepackage{babel}


\begin{document}


    \begin{romanpages}

    \maketitle
    \begin{abstract}
        The graph of a polytope is the graph whose vertex set is the set of vertices of the polytope, and whose edge set is the set of edges of the polytope. Several problems concerning graphs of polytopes are discussed.  The primary result is a set of bounds (Theorem \ref{Thm:Anticliques}) on the maximal size of an anticlique (sometimes called a coclique, stable set, or independent set) of the graph of a polytope based on its dimension and number of vertices.

        Two results concerning properties preserved by certain operations on polytopes are presented.  The first is that the Gale diagram of a join of polytopes is the direct sum of the Gale diagrams of the polytopes and dually, that the Gale diagram of a direct sum of polytopes is the join of their Gale diagrams (Theorem \ref{Thm:DSumAndJoin}).  The second is that if two polytopes satisfy a weakened form of Gale's evenness condition, then so does their product (Theorem \ref{Thm:GaleProduct}).

        It is shown, by other means, that, with only two exceptions, the complete bipartite graphs are never graphs of polytopes (Theorem \ref{Thm:CompBip}). The techniques developed throughout are then used to show that the complete \(3\)-partite graph \(K_{1,n,m}\) is the graph of a polytope if and only if \(K_{n,m}\) is the graph of a polytope (Theorem \ref{Thm:KOneNM}).  It is then shown that \(K_{2,2,3}\) and \(K_{2,2,4}\) are never graphs of polytopes. A conjecture is then stated as to precisely when a complete multipartite graph is the graph of a polytope.  Finally, a section is devoted to results concerning the dimensions for which the graph of a crosspolytope is the graph of a polytope.
    \end{abstract}

    \begin{acknowledgements}
        My sincerest gratitude goes to my advisor, Professor Margaret Bayer.  Without her support and supervision I could not have completed this work.  I would also like to thank Professor Jeremy Martin for the assistance he provided.  They have both been incredibly helpful over the past eight years.  I would also like to thank Professor Heath Martin for his role in shaping my early career in mathematics as well as his recommendation that I attend KU.

        I am grateful to the University of Kansas Department of Mathematics for support over the last eight years, and, in particular, the remaining members of my committee, Jack Porter and Saul Stahl.

        Many thanks go to my friends for listening to my ramblings.  In particular, I would like to thank Lucas Chaffee, Alexander Console, Thomas Enkosky, Jarod Hart and Brandon Humpert.  Special thanks to Robert Bradford who made several insightful suggestions.

        My parents, Bill and Elaine deserve copious thanks for putting up with me for more years than they had to, and for all of their optimism.  Further thanks go to my parents- and brother-in-law, Jack, Robin and Jake.  Finally, my wife, Erin, thank you for your unyielding support, devotion, and faith.
    \end{acknowledgements}
    \tableofcontents{}

    \listoffigures

    \listoftables

    \end{romanpages}

    \include{Chapter_Background/Ch_Background}
    \include{Chapter_Graphs/Ch_Graphs}
    \include{Chapter_Gale_Trans/Ch_Gale_Trans}
     \chapter{Modifying Gale Diagrams}\label{chap:GaleDiagMod}

This chapter explores certain operations on Gale diagrams and how these operations affect the polytopes.



\section{Joins and Direct Sums}

Recall, from Sections \ref{SSec:Join} and \ref{SSec:DirectSum}, the definitions of join and direct sum for point sets.  If \(X=\seta{\ve p_1,\ve p_2\dc\ve p_n}\sbset\R{d_1}\) and \(Y=\seta{\ve q_1,\ve q_2\dc\ve q_m}\sbset\R{d_2}\), then
    \[
        X\join Y
            =
                \setb{\begin{bmatrix}\ve p_i\\\ve0_{d_2}\\-1\end{bmatrix}}{i\in\brac n}
                \cup
                \setb{\begin{bmatrix}\ve 0_{d_1}\\\ve q_j\\1\end{bmatrix}}{j\in\brac m}
            \sbset
                \R{d_1+d_2+1}
    \]
and
    \[
        X\oplus Y
            =
                \setb{\begin{bmatrix}\ve p_i\\\ve0_{d_2}\end{bmatrix}}{i\in\brac n}
                \cup
                \setb{\begin{bmatrix}\ve 0_{d_1}\\\ve q_j\end{bmatrix}}{j\in\brac m}
            \sbset
                \R{d_1+d_2}.
    \]
Since these operations are defined for general point sets, it makes sense to ask what happens if you perform them on two Gale diagrams.
\begin{Theorem}\label{Thm:DSumAndJoin}
    If \(P\) and \(Q\) are polytopes, \(\Gamma\in\galed(P)\), and \(\Lambda\in\galed(Q)\), then
        \[
            \Gamma\oplus \Lambda
                \in
                \galed(P\join Q)
        \]
    and
        \[
            \Gamma\join \Lambda
                \in
                \galed(P\oplus Q).
        \]
\end{Theorem}
\begin{proof}
    Suppose the following:
        \begin{align*}
            \dim P&=d_1;&
                \dim Q&=d_2;\\
            P&\sbset\R{d_1};&
                Q&\sbset\R{d_2};\\
            \vrt P&=\seta{\ve v_1,\ve v_2\dc\ve v_n};&
                \vrt Q&=\seta{\ve w_1,\ve w_2\dc\ve w_m};\\
            \ve 0_{d_1}&\in\relint P;&
                \ve 0_{d_2}&\in\relint Q;\\
            \dim\aff&\seta{\ve v_1,\ve v_2\dc\ve v_{d_1+1}}=d_1;&
                \dim\aff&\seta{\ve w_1,\ve w_2\dc\ve w_{d_2+1}}=d_2;
        \end{align*}
    and
        \begin{align*}
            \rref
                    \begin{bmatrix}
                        1       &   1       &   \cdots  &   1       \\
                        \ve v_1 &   \ve v_2 &   \cdots  &   \ve v_n
                    \end{bmatrix}
                        &=
                        \left[
                            \begin{array}{@{}c|c@{}}
                                I_{d_1+1} & A
                            \end{array}
                        \right]; &
                \rref
                    \begin{bmatrix}
                        1       &   1       &   \cdots  &   1       \\
                        \ve w_1 &   \ve w_2 &   \cdots  &   \ve w_m
                    \end{bmatrix}
                        &=
                        \left[
                            \begin{array}{@{}c|c@{}}
                                I_{d_2+1} & B
                            \end{array}
                        \right].
        \end{align*}
    Then order the vertices of \(P\join Q\) as follows:
        \begin{align*}
                \seta{
                \begin{bmatrix} \ve v_1\\       \ve 0_{d_2}\\   -1  \end{bmatrix}\dc
                \begin{bmatrix} \ve v_{d_1+1}\\ \ve 0_{d_2}\\   -1  \end{bmatrix},
                \begin{bmatrix} \ve 0_{d_1}\\   \ve w_1 \\      1   \end{bmatrix}\dc
                \begin{bmatrix} \ve 0_{d_1}\\   \ve w_{d_2+1}\\ 1   \end{bmatrix},
                \begin{bmatrix} \ve v_{d_1+2}\\ \ve 0_{d_2}\\   -1  \end{bmatrix}\dc
                \begin{bmatrix} \ve v_{n}\\     \ve 0_{d_2}\\   -1  \end{bmatrix},
                \begin{bmatrix} \ve 0_{d_1}\\   \ve w_{d_2+2}\\ 1   \end{bmatrix}\dc
                \begin{bmatrix} \ve 0_{d_1}\\   \ve w_{m}\\     1   \end{bmatrix}
                }.
        \end{align*}
    Here, the first \(d_1+d_2+2=\dim(P\join Q)+1\) vertices are affinely independent.  Now, use the techniques of Section \ref{SSec:ComputingGD} to compute a Gale transformation of \(P\join Q\) as follows:
    Form the matrix
        \begin{align*}
            Z=
            \left[
            \begin{array}{ccc|ccc|ccc|ccc}
                1               &\cdots  &1               &1               &\cdots  &1
                &1              &\cdots  &1               &1               &\cdots  &1   \\
                \ve v_1         &\cdots  &\ve v_{d_1+1}   &\ve 0_{d_1}     &\cdots  &\ve 0_{d_1}
                &\ve v_{d_1+2}  &\cdots  &\ve v_n         &\ve 0_{d_1}     &\cdots  &\ve 0_{d_1}     \\\hline
                \ve 0_{d_2}     &\cdots  &\ve 0_{d_2}     &\ve w_1         &\cdots  &\ve w_{d_2+1}
                &\ve 0_{d_2}    &\cdots  &\ve 0_{d_2}     &\ve w_{d_2+1}   &\cdots  &\ve w_m         \\
                -1              &\cdots  &-1              &1               &\cdots  &1
                &-1             &\cdots  &-1              &1               &\cdots  &1
            \end{array}
            \right].
        \end{align*}


    First, add the first row to the last row, and then perform, in the first \(d_1+1\) rows, the operations that transformation the matrix \(\left[\begin{array}{cccc}1&1&\cdots&1\\ \ve v_1&\ve v_2&\cdots&\ve v_n\end{array}\right]\) into \(\left[\begin{array}{@{}c|c@{}}I_{d_1+1} & A\end{array}\right]\).  Note that in the \(1,2\) and \(1,4\) blocks each column will be the same.  Call this common \((d_1+1)\)-dimensional vector \(\ve v\).  Further, denote an \(r\times s\) matrix with all entries \(0\) by \(\zmat rs\).  That is,
            \begin{comment}
            \rref\left[
                \begin{array}{ccc|ccc|ccc|ccc}
                    1               &\cdots     &1
                    &1              &\cdots     &1
                    &1              &\cdots     &1
                    &1              &\cdots     &1
                    \\
                    \ve v_1         &\cdots     &\ve v_{d_1+1}
                    &\ve0_{d_1}     &\cdots     &\ve0_{d_1}
                    &\ve v_{d_1+2}  &\cdots     &\ve v_n
                    &\ve0_{d_1}     &\cdots     &\ve0_{d_1}
                    \\  \hline
                    \ve 0_{d_2}     &\cdots     &\ve 0_{d_2}
                    &\ve w_1        &\cdots     &\ve w_{d_2+1}
                    &\ve0_{d_2}     &\cdots     &\ve 0_{d_2}
                    &\ve w_{d_2+2}  &\cdots     &\ve w_{m}
                    \\
                    -1              &\cdots     &-1
                    &1              &\cdots     &1
                    &-1             &\cdots     &-1
                    &1              &\cdots     &1
                \end{array}
            \right]\\
            \end{comment}
        \begin{align*}
            M&= \rref Z
                =
                \rref\left[
                    \begin{array}{c|c|c|c}
                        I_{d_1+1}
                            &\begin{array}{ccc}{\olay{\ve v}{\ve w_1}}&\cdots&{\olay{\ve v}{\ve w_{d_2+1}}}\end{array}
                            &A
                            &\begin{array}{ccc}\olay{\ve v}{\ve w_{d_2+2}}&\cdots&\olay{\ve v}{\ve w_m}\end{array}
                        \\ \hline
                        \zmat{d_2+1}{d_1+1}
                            &\begin{array}{ccc}\ve w_1&\cdots&\ve w_{d_2+1}\\ 2&\cdots&2\end{array}
                            &\zmat{d_2+1}{n-d_1+1}
                            &\begin{array}{ccc}\ve w_{d_2+2}&\cdots&\ve w_{m}\\ 2&\cdots&2\end{array}
                    \end{array}
                \right].
        \end{align*}

    Next, use the bottom row of the matrix to turn the \(1,2\) and \(1,4\) blocks into all zeros.  Then move the bottom row to the \((d_1+2)\)th row, divide it by \(2\) and shift each of the remaining rows down, that is:
        \begin{align*}
            M
                &=
                \rref\left[
                    \begin{array}{c|c|c|c}
                        I_{d_1+1}
                            &\zmat{d_1+1}{d_2+1}
                            &A
                            &\zmat{d_1+1}{m-d_2-1}
                        \\ \hline
                        \zmat{d_2+1}{d_1+1}
                            &\begin{array}{ccc}1&\cdots&1\\ \ve w_1&\cdots&\ve w_{d_2+1}\end{array}
                            &\zmat{d_2+1}{n-d_1+1}
                            &\begin{array}{ccc}1&\cdots&1\\ \ve w_{d_2+2}&\cdots&\ve w_{m}\end{array}
                    \end{array}
                \right].
        \end{align*}

    Now, in the bottom \(d_2+1\) rows, perform the row operations that transformation the matrix \(\left[\begin{array}{cccc}1&1&\cdots&1\\ \ve w_1&\ve w_2&\cdots&\ve w_m\end{array}\right]\) into \(\left[\begin{array}{@{}c|c@{}}I_{d_2+1} & B\end{array}\right]\).  Thus
        \begin{align*}
            M
                &=
                \left[
                    \begin{array}{c|c}
                        I_{d_1+d_2+2}
                            &\hspace{-5pt}\begin{array}{c|c}
                                A   &[0]\\
                                \hline
                                [0] &B
                             \end{array}
                    \end{array}\hspace{-5pt}
                \right]
        \end{align*}
    and so a Gale transformation of \(P\join Q\) is given by the multiset of the columns of the matrix
        \begin{align*}
            \left[
                \begin{array}{c|c|c|c}
                    -\Tr A  &[0]    &I      &[0]    \\ \hline
                    [0]     &-\Tr B &[0]    &I
                \end{array}
            \right].
        \end{align*}
    These are exactly the elements of \(\Gamma\oplus\Lambda\).

    The second result holds by oriented matroid duality.
\end{proof}


\subsection{\protect$d\protect$-Polytopes with \protect$d+2\protect$ Vertices}\label{SSec:dPlusTwo}
    As an application of Theorem \ref{Thm:DSumAndJoin}, all \(d\)-polytopes with \(d+2\) vertices are classified.  This section follows the discussion found in \cite{McMullenBook}.

    Let \(P\) be a \(d\)-polytope with \(d+2\) vertices, and let \(\Gamma\) be a standard Gale diagram of \(P\).  Since \(\Sp0=\seta{-1,1}\sbset\RR\), as a set \(\seta{-1,1}\sbset\Gamma\sbset\seta{-1,0,1}\).  Because \(\Gamma\) is a Gale diagram,
        \begin{align*}
            p
                &=  \card{\setb{\ve x\in\Gamma}{\ol{\ve x}=-1}}
                    \ge 2\\
            q
                &=  \card{\setb{\ve x\in\Gamma}{\ol{\ve x}=1}}
                    \ge 2.
        \end{align*}
    This leaves \(a=d+2-p-q\) points in \(\Gamma\), and they must all be at the origin.  A point at the origin in a Gale diagram corresponds to the apex of a pyramid in the polytope (Theorem \ref{Thm:GalePyr}).  Thus \(P\) is an \(a\)-fold pyramid over a polytope \(Q\) whose Gale diagram \(\Lambda\) has \(p\ge 2\) points on one side of the origin, and \(q\ge 2\) points on the other.

    The Gale diagram \(\Lambda\) is the join of the Gale diagrams of a \((p-1)\)-simplex and a \((q-1)\)-simplex.  Hence, by Theorem \ref{Thm:DSumAndJoin}, \(P\) is combinatorially equivalent to the polytope
        \[
            \pyr_a\left(\simp{p-1}\oplus\simp{q-1}\right).
        \]
    where \(a+p+q=d+2\) and \(p,q\ge 2\).

    Note that the combinatorial type of each \(d\)-polytope with \(d+2\) vertices can be specified by the unordered pair \((p,q)\).  So to count the number of combinatorial types of these polytopes, one only needs to count the number of such pairs.  Assume, without loss of generality, that \(p\le q\).  The pairs are then of the following form:
        \begin{align*}
            (2,q)
                &,\qquad 2\le q\le d\\
            (3,q)
                &,\qquad 3\le q\le d-1\\
                &\vdots\\
            \left(\floor{\frac d2},q\right)
                &,\qquad \floor{\frac d2}\le q\le\ceil{\frac d2}.
        \end{align*}

    Thus, if \(d\) is odd, then there are
        \begin{align*}
            2+4+\dotsb+(d-1)
                &=  2\left(1+2+\dotsb+\frac{d-1}2\right)
                =   \frac{d^2}4-\frac14
                =   \floor{\frac{d^2}4}
        \end{align*}
    combinatorial types of these polytopes.  If \(d\) is even, then there are
        \begin{align*}
            1+3+\dotsb+(d-1)
                &=  2+4+\dotsb+d-\frac d2
                =   \frac{d^2}4
                =   \floor{\frac{d^2}4}
        \end{align*}
    types of these polytopes.

    In \cite{McMullenBook}, the following lemma is proved.
    \begin{Lemma}
        Let \(\Gamma\) be a Gale diagram of a \(d\)-polytope \(P\) with \(n\) vertices.  Then \(P\) is a simplicial polytope if and only if for every hyperplane \(H\in\R{n-d-1}\) containing \(\ve 0\) the following holds:
            \[
                \ve 0
                    \notin \relint\conv(\Gamma\cap H).
            \]
    \end{Lemma}

    Combining this with the above discussion yields:
    \begin{Theorem}
        There are \(\floor{{d^2}/4}\) combinatorial types of \(d\)-polytopes with \(d+2\) vertices.  Furthermore, each of these polytopes is of the form \(\pyr_a\left(\simp{p-1}\oplus\simp{q-1}\right)\) where \(a+p+q=d+2\) and \(p,q\ge2\).  Moreover, \(\floor{d/2}\) of these polytopes are simplicial.
    \end{Theorem}

    \begin{comment}
    Thus, letting \(\Gamma(\simp{j})\) denote the standard Gale diagram of the \(j\)-simplex,
         \[
            \Gamma  =   \Gamma(\simp{k_{-1}-1})\join\Gamma(\simp{k_{1}-1}).
         \]
    Hence, by Theorem \ref{Thm:DSumAndJoin}, \(P\) is a direct sum of two simplices.
    \end{comment}

\section{Adding Points to Gale Diagrams}
    The first thing one should do in a section entitled ``Adding Points to Gale Diagrams" is discuss under what conditions adding a point to a Gale diagram yields a Gale diagram.  Thus, suppose \(\Gamma\in\galed(P)\) is a Gale diagram of some \(d\)-polytope \(P\) with \(n\) vertices, and let \(\ve x\in\R{n-d-1}\) be any point.  Then \(\Gamma\cup\seta{\ve x}\) is a Gale diagram of some \((d+1)\)-polytope with \(n+1\) vertices.  This is true since every hyperplane still has at least two points in both of its open half-spaces.  It should be clear that the location of the new point does matter in determining the new polytope.  Adding a point to a Gale diagram increases the number of points by \(1\), but not the dimension of the space in which the Gale diagram lies. Thus the dimension of the new polytope is \(1\) greater than the dimension of the old polytope.

    Points can be added to Gale diagrams in many ways.  Unfortunately, it is possible given two Gale diagrams  \(\Gamma,\Lambda\in\galed(P)\) to add a point to both \(\Gamma\) and \(\Lambda\) in the same way (such as adding a point antipodal to the point corresponding to a particular vertex), yet not have the two new polytopes be combinatorially equivalent.

    \subsection{Adding an Antipode to a Gale Diagram}
        \begin{figure}[p!hbt]
            \centering
                \includegraphics[width=.7\textwidth, page=16]{pictures.pdf}
            \caption{Two Gale diagrams of $\xp3$.\label{Fig:2GaleDXp3}}
        \end{figure}

        \begin{figure}[p!hbt]
            \centering
                \includegraphics[width=.8\textwidth, page=17]{pictures.pdf}
            \caption{Adding an antipode $\ol{1'}$ to the point $\ol{1}$ in a Gale diagram of $\xp3$. (cf.{} Figure \ref{Fig:AddPt2Gale2})\label{Fig:AddPt2Gale1}}
        \end{figure}

        \begin{figure}[p!hbt]
            \centering
                \includegraphics[width=.8\textwidth, page=18]{pictures.pdf}
            \caption{Adding an antipode $\ol{1'}$ to the point $\ol{1}$ in a Gale diagram of $\xp3$. (cf.{} Figure \ref{Fig:AddPt2Gale1})\label{Fig:AddPt2Gale2}}
        \end{figure}

        As an example; consider the crosspolytope \(\xp3\).  Figure \ref{Fig:2GaleDXp3} shows two consubstantial standard Gale diagrams for \(\xp3\).  If a point \(\ol{1'}\) is added that is antipodal to \(\ol 1\) in each of these Gale diagrams, then Figures \ref{Fig:AddPt2Gale1} and \ref{Fig:AddPt2Gale2} show the graphs of the polytopes with each of these Gale diagrams.  Let \(P\) be a polytope with the Gale diagram in Figure \ref{Fig:AddPt2Gale1}, and \(Q\) be a polytope with the Gale diagram in Figure \ref{Fig:AddPt2Gale2}.  Note that \(\conv\seta{2,5}\) is not an edge in \(P\), whereas in \(Q\) it is.  Hence, \(P\) and \(Q\) cannot be combinatorially equivalent.

        The facets of \(P\) are the \(3\)-dimensional pyramids \(23456\) and \(12345\) each with base \(2345\), as well as the following \(3\)-simplices.
            \begin{align*}
                &11'45&     &11'35&     &11'24&     &11'23\\
                &1'456&     &1'356&     &1'246&     &1'236
            \end{align*}
        The facets of \(Q\) are the triangular bipyramid \(23456\) with base \(256\), as well as the following \(3\)-simplices.
            \begin{align*}
                &11'45&     &11'35&     &11'24&     &11'23\\
                &1'456&     &1'356&     &1'246&     &1'236\\
                &1245&      &1235&      &&&
            \end{align*}
    \subsection{Duplicating a Point in a Gale Diagram}
        On the other hand, an operation that is well defined is that of adding a duplicate of an existing point to a Gale diagram.
        Let \(P\) be a \(d\)-polytope with vertex set \(\vrt P=\seta{\ve v_1,\ve v_2\dc\ve v_n}\).  Let \(\Gamma\) be a Gale diagram of \(P\), and let \(Q\) be a \((d+1)\)-polytope with vertex set \(\vrt Q=\seta{\ve w_{1'},\ve w_1,\ve w_2\dc\ve w_n}\) whose Gale diagram \(\Lambda\) satisfies \(\ol{\ve w}_{1'}=\ol{\ve v}_1\) and \(\ol{\ve w}_i=\ol{\ve v}_i\) otherwise.

        Let \(Y\) be a cofacet of \(Q\), that is, a minimal coface and consider the following four cases:
            \begin{enumerate}
                \item   (\(\seta{\ve w_1,\ve w_{1'}}\sbset Y\))  This case cannot happen since the set \(\ol{Y}\setminus\seta{\ol{\ve w}_1}\) captures the origin, and \(Y\setminus\seta{\ve w_1}\varsubsetneq Y\).
                \item   (\(\ve w_1\in Y\) and \(\ve w_{1'}\notin Y\))  For this case, let \(X=\setb{\ve v_i}{\ve w_i\in Y}\).  Then since \(\ol Y=\ol X\), the set \(Y\) is a cofacet if and only if \(X\) is a cofacet.
                \item   (\(\ve w_1\notin Y\) and \(\ve w_{1'}\in Y\))  Here, set \(X=\setb{\ve v_i}{\ve w_i\in Y\text{ or }i=1}\).  In this case, \(\ol Y=\ol X\) again, and so \(Y\) is a cofacet if and only if \(X\) is a cofacet.
                \item   (\(\seta{\ve w_1,\ve w_{1'}}\cap Y=\mt\)) For the final case, again set \(X=\setb{\ve v_i}{\ve w_i\in Y}\).  Once more, \(\ol Y=\ol X\), so that \(Y\) is a cofacet if and only if \(X\) is a cofacet.
            \end{enumerate}
        Thus the cofacets (and hence the facets) of \(Q\) are determined by the cofacets (and hence the cofacets) of \(P\) and vice versa. Geometrically, \(Q\) is combinatorially equivalent to a polytope with the following vertices:
            \begin{align*}
                \ve w_1
                    &=   \begin{bmatrix}\ve v_1\\1 \end{bmatrix}&
                \ve w_{1'}
                    &=   \begin{bmatrix}\ve v_1\\-1 \end{bmatrix}&
                \ve w_j
                    &=   \begin{bmatrix}\ve v_j\\0 \end{bmatrix},\,j\in\brac n\setminus\seta1
            \end{align*}
        Conceptually, first place \(P\) in \(\R{d+1}\).  Then the vertex \(\ve v_1\) is replaced by a line segment whose relative interior passes through \(\ve v_1\), and whose affine hull intersects that of the original polytope at precisely the point \(\ve v_1\).

        If \(P\) is a \(2\)-polytope with \(n\) vertices, then performing this operation at any vertex yields a polytope that is combinatorially equivalent to performing the operation at any other vertex.  The resulting polytope is combinatorially equivalent to the cyclic polytope \(\cyc{n+1}3\).  See Figure \ref{Fig:DupGale} for an example.  This process is called a \dfn{simplicial wedge} in \cite{SimplicialWedge}.

        \begin{figure}[p!hbt]
            \centering
                \includegraphics[width=.7\textwidth, page=32]{pictures.pdf}
            \caption{Duplicating the point\protect$\ol1\protect$ in a Gale diagram of \protect$\cyc25\protect$.\label{Fig:DupGale}}
        \end{figure} 
    \include{Chapter_Gale/Ch_Gale}
    \chapter{Anticliques in Graphs of Polytopes}
\label{chap:Anticliques}

Let \(G=(V, E)\) be a graph.  An \dfn{anticlique} is a subset \(A\) of \(V\) (of cardinality at least \(2\)) such that \(E\cap\binom A2=\mt\).  If \(\card A=k\), then \(A\) is said to be a \(k\)-anticlique.  A \(2\)-anticlique is simply a nonedge.

Recall that if \(G\) is the graph of a \(d\)-polytope \(P\) with \(n\) vertices, and \(\Gamma\in\galed(P)\) is a Gale diagram of \(P\), then \(\seta{{\ve v}_i,{\ve v}_j}\) is a nonedge of \(P\) if and only if there is some hyperplane \(H_{i,j}\sbset\R{n-d-1}\) containing \(\ve0\) such that \(H_{i,j}^{(+)}\cap\Gamma=\seta{\ol{\ve v}_i,\ol{\ve v}_j}\).  Such a hyperplane is called a \dfn{separating} hyperplane.  The normal vector to \(H_{i,j}\) which has positive inner product with \(\ol{\ve v}_i,\ol{\ve v}_j\) and norm \(1\) is denoted \(\ve x_{i,j}\).

If \(P\) is a \(d\)-polytope with \(d+k\) vertices, then Theorem \ref{Thm:ConnectivityOfGraph} in Section \ref{Sec:GraphsOfPolytopes} implies that each vertex of \(\gr P\) must have degree at least \(d\).  If a graph \(H\) with \(d+k\) vertices has a \((k+1)\)-anticlique, then each vertex in the anticlique has degree at most \(d-1\), and hence \(H\) is not \(d\)-realizable.  However, the graph \(\left(\brac{d+k},\binom{\brac{d+k}}2\setminus\binom{\brac k}2\right)\) is \(d\)-connected, and furthermore has a \(K_{d+1}\) minor (the induced subgraph on the vertex set \(\seta{k, k+1\dc k+d}\) is complete) and thus \emph{could} be the graph of a \(d\)-polytope.  It will be shown that this cannot happen.  More precisely for \(k\ge 2\), let
    \[
        f(d,k)=\max\setb{n}{\text{there is some \(d\)-polytope \(P\) with \(d+k\) vertices and an \(n\)-anticlique}}.
    \]
Then the goal is to show that \(f(d,k)<k\) for \(k>2\).

\section{An Upper Bound on \protect$f\protect$}
First note that \(f\) is weakly increasing in both arguments.

\begin{Theorem}\ph
    \begin{enumerate}
        \item   \(f(d,k)\le f(d+1,k)\)
        \item   \(f(d,k)\le f(d,k+1)\)
    \end{enumerate}
\end{Theorem}
\begin{proof}
    Let \(P\) be a \(d\)-polytope with \(d+k\) vertices, and an \(f(d,k)\)-anticlique.
        \begin{enumerate}
            \item   The \((d+1)\)-polytope \(\pyr(P)\) has \((d+k)+1=(d+1)+k\) vertices and an \(f(d,k)\)-anticlique.
            \item   Let \(F\) be any facet of \(P\).  The \(d\)-polytope \(K(P;F)\) (defined in section \ref{Subsec:Kleetopes}) has \(d+(k+1)\) vertices and an \(f(d,k)\)-anticlique.
        \end{enumerate}
\end{proof}



Recall that a standard Gale diagram is one in which each point is either on the unit sphere, or at the origin.  Further if a point in the Gale diagram is at the origin, then the polytope is a pyramid with apex the corresponding point.  Since the main question is concerned with sizes of anticliques, and the anticliques of a pyramid are exactly those of its base, nothing is lost or gained in the assumption that a polytope is not a pyramid.

\begin{Theorem}\label{Thm:TrivialitiesAnticliques}
    Let \(P\) be a \(d\)-polytope with vertex set \(\seta{\ve v_1,\ve v_2\dc\ve v_{n}}\);  \(A=\seta{i_1,i_2\dc i_k}\sbset\brac n\); \(\setb{\ve v_{i}}{i\in A}\) be an anticlique;  \(\Gamma\) be a standard Gale diagram of \(P\); and \(\seta{i,j}\sbset A\).  Then
        \begin{enumerate}
            \item   \(\ol{\ve v}_i\ne\ve 0\).
            \item   \(\ol{\ve v}_i\ne-\ol{\ve v}_j\).
        \end{enumerate}
    Furthermore if \(k\ge3\), then
        \begin{enumerate}[resume]
            \item\label{ThmPt:notequal}   if \(\ol{\ve v}_i=\ol{\ve v}_j\) , then \(i=j\).
        \end{enumerate}
\end{Theorem}
\begin{proof}\ph
    \begin{enumerate}
        \item   If \(\ol{\ve v}_i=\ve 0\), then \(\conv\seta{\ve v_i,\ve v_t}\) is a face of \(P\) for each \(t\in\brac n\setminus\{i\}\).  In particular, \(\conv\seta{\ve v_i,\ve v_j}\) is an edge of \(G(P)\).
        \item  If \(\ol{\ve v}_i=-\ol{\ve v}_j\),and \(H_{i,j}\) is a separating hyperplane for \(\ol{\ve v}_i,\ol{\ve v}_j\) with normal vector \(\ve x_{i,j}\), then
                \[
                    0<\ip{\ve x_{i,j}}{\ol{\ve v}_i}=-\ip{\ve x_{i,j}}{\ol{\ve v}_j}<0.
                \]
        \item  Let \(H_{a,b}\) be a separating hyperplane for \(\ol{\ve v}_a,\ol{\ve v}_b\) with normal \(\ve x_{a,b}\) for each pair \(a,b\in A\).  Suppose \(\ol{\ve v}_i=\ol{\ve v}_j\) with \(i\ne j\).  Then for \(r\in A\setminus\seta{i,j}\),
                \[
                    0   <\ip{\ve x_{j,r}}{\ol{\ve v}_j}
                        =\ip{\ve x_{j,r}}{\ol{\ve v}_i}
                        \le 0.
                \]
            Hence \(i=j\).\qedhere
    \end{enumerate}
\end{proof}
\begin{Theorem}
    \(f(d,2)=2\)
\end{Theorem}
\begin{proof}
    If \(P\) is a \(d\)-polytope with \(d+2\) vertices, then \(P\) has a \(1\)-dimensional Gale diagram.  Since a nonedge requires that there be a separating hyperplane, and in this case there is only one possible hyperplane (namely the origin), each vertex can have at most one nonneighbor.  This implies that \(f(d,2)\le2\).  The polytope \(\pyr_{d-2}(\xp2)\) is \(d\)-dimensional with \(d+2\) vertices and a \(2\)-anticlique.  Thus \(f(d,2)=2\).
\end{proof}
\begin{Theorem}
    \(f(d,3)<3\)
\end{Theorem}
\begin{proof}
    Let \(P\) be a \(d\)-polytope with \(d+3\) vertices, and let \(\Gamma\) be  a standard Gale diagram of \(P\).  Suppose that \(P\) has a \(3\)-anticlique \(\seta{\ve v_1,\ve v_2,\ve v_3}\), and let \(\ve v_4\) be any other vertex of \(P\).  Note that since \(\Gamma\sbset\Sp1\) (assuming again that \(P\) is not  a pyramid), by writing each \(\ol{\ve v}_i\in\Gamma\) in polar coordinates only the angle \(\varphi_i\in[0,2\pi)\) is necessary to specify \(\ol{\ve v}_i\).  By Theorem \ref{Thm:TrivialitiesAnticliques} part \ref{ThmPt:notequal}, none of \(\varphi_1,\varphi_2,\varphi_3\) are equal, so assume \(\varphi_1<\varphi_2<\varphi_3\).  Further, if \(\varphi_1>0\), then rotating each point in \(\Gamma\) clockwise by an angle of \(\varphi_1\) produces a consubstantial standard Gale diagram, so assume
        \[
            0=\varphi_1<\varphi_2<\varphi_3<2\pi.
        \]

        \begin{center}
            \begin{figure}[h!bt]
                \includegraphics[page=23, width=.3\textwidth]{pictures.pdf}
            \end{figure}
        \end{center}

    Since \(\ve v_1\ve v_2\) is not an edge of \(P\), there is some hyperplane \(H_3\) through the origin such that \(H_3^{(+)}\cap\Gamma=\seta{\ol{\ve v}_1,\ol{\ve v}_2}\).  Therefore, \(\varphi_4\notin[0,\varphi_2]\).  Similarly, \(\varphi_4\notin[\varphi_2,\varphi_3]\), and \(\varphi_4\notin[\varphi_3,2\pi)\).  Since \(\Gamma\) is a standard Gale diagram, this means that \(\ol{\ve v}_4=\ve0\).  But this contradicts the assumption that \(P\) is not a pyramid.
\end{proof}


\begin{Lemma}\label{Lem:LocalCone}
    If
        \begin{itemize}
            \item   \(P\) is a \(d\)-polytope with vertex set \(V\cup Y\) of cardinality \(d+k\),
            \item   \(\card V=d\) and \(\card Y=k\),
            \item   \(Y\) is an anticlique of \(\gr P\), and
            \item   \(\ve v\in V\), then
        \end{itemize}
    \(\ve v\ve y\) is an edge of \(\gr P\) for each \(y\in Y\).
\end{Lemma}
\begin{proof}
    Let \(\ve v\in V\), \(\ve y\in Y\).  Since \(P\) is a \(d\)-polytope, it is \(d\)-connected, therefore each vertex of \(G(P)\) is adjacent to at least \(d\) other vertices.  But since \(Y\) is an anticlique, \(\ve y\) is adjacent to at most \(d\) other vertices. Hence \(\ve y\) is adjacent to exactly \(d\) other vertices, and since none of these other vertices is in \(Y\), \(\ve v\ve y\) is an edge of \(G(P)\).
\end{proof}

    The following is a general fact about polytopes.  A proof is included for completeness, but
\begin{Lemma}\label{Lem:LocallySimple}
    Let \(P\) be a \(d\)-polytope, \(\ve y\in\vrt P\) with \(\deg\ve y=d\), and \(V=N(\ve y)=\seta{\ve v_1,\ve v_2\dc\ve v_d}\).  Then for each subset \(X\) of \(V\) with cardinality \(d-1\) there is a unique facet \(F\) of \(P\) with \(X\sbset\vrt F\).
\end{Lemma}
\begin{proof}
    Each vertex of \(P\) lies in at least \(d\) facets since \(P\) is a \(d\)-polytope.  Let \(F\) be a facet containing \(\ve y\).  Since \(F\) is a \((d-1)\)-polytope, each vertex of \(G(F)\) has degree at least \(d-1\).  Thus \(\ve y\) must have either \(d-1\) or \(d\) neighbors in \(F\).  If \(\ve y\) had each of its \(d\) neighbors in \(F\), then \(\ve y\) would only lie in one facet.  Thus each facet in which \(\ve y\) lies contains exactly \(d-1\) neighbors of \(\ve y\).  Further, each element of \(\binom V{d-1}\) determines a unique facet of \(P\) since \(\ve y\) lies in at least \(d\) facets.
\end{proof}

Notice that this argument also shows that the set \(V\) above is affinely independent, and that \(V\) is not contained in any proper face of \(P\).

\begin{Theorem}\label{Thm:Anticliques}
    If \(k>2\), then \(f(d,k)<k\).
\end{Theorem}
\begin{proof}
    Proceed by induction on \(d\).  For the base case: the graph of a \(2\)-polytope with \(n\) vertices is a cycle of length \(n\), so a maximal anticlique can be obtained by taking every other vertex. Therefore \(f(2,k)=1+\floor{k/2}<k\).  Thus, suppose  for some \(d\) that \(f(d-1,k)<k\) for all \(k>2\).  The proof of the contrapositive will be shown, that is, if \(f(d,k)=k\), then \(k=2\).

    Let \(P\) be a \(d\)-polytope with \(d+k\) vertices and, without loss of generality, suppose that \(P\) is not a pyramid.  Suppose further, that \(P\) has a \(k\)-anticlique \(Y=\seta{\ve y_1,\ve y_2\dc\ve y_k}\).  Write \(\vrt(P)=\seta{\ve v_1,\ve v_2\dc\ve v_d}\cup Y\), let \(V=\seta{\ve v_1,\ve v_2\dc\ve v_d}\), and \(V_i=V\setminus\seta{\ve v_i}\) for \(i\in\brac d\).  Fix some \(i\in\brac d\), and for each \(j\in\brac k\), let \(F_{i,j}\) be the facet of \(P\) which contains \(V_i\cup\seta{\ve y_j}\) (Lemmata \ref{Lem:LocalCone} and \ref{Lem:LocallySimple}), and \(G_i\) be the smallest face of \(P\) which contains \(V_i\).

    Notice that \(G_i\sbset F_{i,j}\) for each \(j\in\brac k\), and that \(G_i\) is either a facet or a ridge since \(V\) is an affinely independent set.

        \begin{enumerate}
            \item   (\(G_i\) is a facet.)  In this case, for each \(j\in\brac k\), \(G_i=F_{i,j}=\conv(Y\cup V_i)\).  Thus \(P\) is a pyramid over \(G_i\) with apex \(\ve v_i\).  This is a contradiction.
            \item   (\(G_i\) is a ridge.)  Let \(F_1,F_2\) be the facets containing \(G_i\).  In this case, \(Y\) can be partitioned into two nonempty sets
                    \begin{align*}
                        Y_1 &=  \seta{\ve y_{a_1},\ve y_{a_2}\dc\ve y_{a_r}}&
                        Y_2 &=  \seta{\ve y_{b_1},\ve y_{b_2}\dc\ve y_{b_s}}&
                    \end{align*}
                such that
                    \begin{align*}
                        Y_1 &\sbset F_{i,a_1}   =   F_1&
                        Y_2 &\sbset F_{i,b_1}   =   F_2.&
                    \end{align*}

                If \(\card{Y_1}=2\), then \(F_1\) is a \((d-1)\)-polytope with \((d-1)+2\) vertices and \(G_i\) is a facet of \(F_1\).  Thus a Gale diagram of \(F_1\) is one dimensional with \(d+1\) vertices, and both \(\ol{\ve y}_{a_1}\) and \(\ol{\ve y}_{a_2}\) are on the same side of the origin.  However, \(\ve 0\in\relint\conv{\left(\ol{\vrt(F_1)\setminus Y_1}\right)}\).  Similarly, \(\card{Y_2}\ne2\).

                If \(\card{Y_1}\ge 3\), then \(F_{i,a_1}\) is a \((d-1)\)-polytope with \((d-1)+\card{Y_1}\) vertices, and a \(\card{Y_1}\)-anticlique, that is, \(f(d-1,\card{Y_1})\ge\card{Y_1}\).  This contradicts the inductive hypothesis.

                Hence \(\card{Y_1}=\card{Y_2}=1\), whence \(P\) is a \(d\)-polytope with \(d+2\) vertices and a \(2\)-anticlique.  Thence \(k=2\).
        \end{enumerate}
\end{proof}

\section{A Lower Bound on \protect$f\protect$}

\begin{Lemma}
    If
        \begin{itemize}
            \item   \(P\) is a \(d\)-polytope with \(d+k\) vertices,
            \item   \(\ve v\) is a vertex of \(P\) such that each facet containing \(\ve v\) is a simplex, and
            \item   \(\ve v\) is contained in a \(q\)-anticlique of \(P\), then
        \end{itemize}
    \(f(d,k+i)\ge q+i-1\) for \(i\in\brac d\).
\end{Lemma}
    For example, such a vertex exists when \(P\) is a simplex (in which case each vertex has this property) or if \(P\) is of the form \(P=K(P';F)\) where \(F\) is a facet of \(P'\) which is a simplex, so the Lemma is not vacuous.
\begin{proof}
    Let \(F_1,F_2\dc F_d\) be facets of \(P\) containing \(\ve v\), and set
        \begin{align*}
            P_0(\ve v)
                    &=P\\
            P_i(\ve v)
                    &=K(P;F_1,F_2\dc F_i)\\
            \seta{\ve v_i}
                    &=\vrt(P_i(\ve v))\setminus\vrt(P_{i-1}(\ve v)).
        \end{align*}
    Further, let \(A'\) be a \(q\)-anticlique of \(P\) which contains \(\ve v\), and set \(A=A'\setminus\seta{\ve v}\).

    For a fixed \(i\in\brac d\), the inclusion \(N(\ve v_i)\sbset N(\ve v)\cup\seta{\ve v}\) in \(\gr{P_i(\ve v)}\) implies that  \(A\cup\seta{\ve v_1,\ve v_2\dc\ve v_i}\) is a \((q-i-1)\)-anticlique in \(\gr{P_i(\ve v)}\).
\end{proof}

\begin{Theorem}\label{Thm:Bounds}
    If \(k,d>2\), then \(\ds k-1-\floor{\frac{k-3}d}\le f(d,k)\le k-1\).
\end{Theorem}
\begin{proof}
    Fix \(d>2\) and set \(t=\floor{(k-3)/d}\) and let \(Q_0=K(\simp d;F)\) for some facet \(F\) of \(\simp d\).  Let \(\ve v_0\) be a vertex of \(Q_0\) and \(F_{0,1},F_{0,2}\dc F_{0,d}\) be the facets of \(Q_0\) which contain \(\ve v_0\).  Then for \(k\in\brac{d}\), the polytope \(K(Q_0;F_{0,1},F_{0,2}\dc F_{0,k})\) is a \(d\)-polytope with \(d+2+k\) vertices and a \((k+1)\)-anticlique.  Now, inductively define, for \(n\in\N\setminus\seta0\), the polytope
        \[
            Q_n
                =
                    K(Q_{n-1};F_{n-1,1},F_{n-1,2}\dc F_{n-1,d})
        \]
    where \(\seta{F_{n-1,1},F_{n-1,2}\dc F_{n-1,d}}\) is the set of facets that contain some fixed vertex in a maximal anticlique of \(Q_{n-1}\).

    Finally, for \(k\in\brac{d}\) the polytope \(K(Q_n;F_{n,1},F_{n,2}\dc F_{n,k})\) is a \(d\)-polytope with \(d+(nd+2+k)\) vertices and an anticlique of cardinality \(nd+2+k-n-1\).
\end{proof}

Notice that if \(\floor{(k-3)/d}=0\) (that is, \(k\le d+2\)), then the upper and lower bounds on \(f\) agree.

\section{The Value of \protect$f\protect$ in Dimension \protect$3\protect$}

If \(d=3\), then Theorem \ref{Thm:Bounds} says that \( k-1-\floor{(k-3)/3}\le f(3,k)\le k-1\).  In this case, the lower bound can be rewritten as \(\ds k-1-\floor{(k-3)/3}=\ceil{2k/3}\).

    \subsection{Euler's Theorem}
        Euler's Theorem states that the \(f\)-vector of a \(d\)-polytope lies on a certain hyperplane in \(\R{d}\).  For a proof, see \cite{GrunBook}, \cite{McMullenBook}, or \cite{ZieglerBook}.  All that will be needed is the \(d=3\) case:
            \begin{Theorem}[Euler]
                If \(P\) is a \(3\)-polytope, then \(f_0(P)-f_1(P)+f_2(P)=2\).
            \end{Theorem}
        Euler's Theorem can be proven for planar graphs in general; in this case \(f_2\) counts the number of regions into which the graph divides the plane.

        Since each region of a planar graph is bounded by at least \(3\) edges, and each edge lies on at most \(2\) facets, \(3R\le2\card{E(G)}\) where \(R\) is the number of regions into which the graph divides the plane.  If \(G\) is the graph of a polytope \(P\), then \(R=f_2(P)\).  If the number of edges bounding a region is at least \(4\), then this inequality becomes \(4R\le2\card{E(G)}\), i.e.{}, \(2R\le\card{E(G)}\).  Combining this with Euler's Theorem yields \(\card{E(G)}\le3\card{V(G)}-6\) for a general planar graph, and \(\card{E(G)}\le2\card{V(G)}-4\) for a planar graph with no \(3\)-cycles.  In particular, if \(G\) is bipartite, then it has no \(3\)-cycles.

    \subsection{Bipartite Graphs Induced by Anticliques}

    If \(P\) is a \(3\)-polytope and \(A\) is an anticlique of \(G=\gr P\), then \(A\) induces a bipartite graph \(G_A\) whose vertex set is that of \(G\), and whose edge set is the set of edges in \(G\) which contain a vertex of the anticlique \(A\).  That is:
        \begin{align*}
            V(G_A)
                &=  V(G)\\
            E(G_A)
                &=  \setb{ax}{a\in A}\cap E(G).
        \end{align*}
    Note that \(G_A\) is planar since \(G\) is planar, and \(G_A\) is a subgraph of \(G\).

    \begin{Theorem}
        \(f(3,k)=\ceil{2k/3}\)
    \end{Theorem}
    \begin{proof}
        Suppose that \(P\) is a \(3\)-polytope with \(3+k\) vertices and a \((\ceil{2k/3}+1)\)-anticlique \(A\).  Let \(G=\gr P\) and \(a\in A\).  Since each edge of \(G\) on which \(a\) lies is also an edge of \(G_A\), and \(a\) lies on at least \(3\) edges of \(G\), it follows that
            \begin{align*}
                \card{E(G_A)}
                    &\ge
                        3(\ceil{2k/3}+1)
                    \ge
                        3(2k/3+1)
                    =
                        2k+3
            \end{align*}
        On the other hand, \(G_A\) is a bipartite planar graph with \(3+k\) vertices.  It therefore has at most \(2(3+k)-4=2k+2\) edges.
    \end{proof}




    \begin{table}[h]
    \begin{tabular}{c|ccccccccccccccc}
       \multicolumn{1}{l|}{\backslashbox[0pt][]{d\kern-5em}{\kern-.5em k}}   & \(\vphantom{\ds\sum_a^b}\)2 & 3 & 4 & 5 & 6        & 7        & 8        & 9        & 10       & 11       & 12        & 13        & 14        & 15        & 16        \\\hline
        2 & 2 & 2 & 3 & 3 & 4 & 4 & 5 & 5 & 6 & 6 & 7 & 7 & 8  & 8   & 9 \\
        3 & 2 & 2 & 3 & 4 & 4 & 5 & 6 & 6 & 7 & 8 & 8 & 9 & 10 & 10 & 11 \\
        4 & 2 & 2 & 3 & 4 & 5 & \(\mbf5\) & \(\mbf6\) & \(\mbf7\) & \(\mbf8\) & \(\mbf8\) & \(\mbf9\)  & \(\mbf{10}\) & \(\mbf{11}\) & \(\mbf{11}\) & \(\mbf{12}\) \\
        5 & 2 & 2 & 3 & 4 & 5 & 6 & \(\mbf6\) & \(\mbf7\) & \(\mbf8\) & \(\mbf9\) & \(\mbf{10}\) & \(\mbf{10}\) & \(\mbf{11}\) & \(\mbf{12}\) & \(\mbf{13}\) \\
        6 & 2 & 2 & 3 & 4 & 5 & 6 & 7 & \(\mbf7\) & \(\mbf8\) & \(\mbf9\) & \(\mbf{10}\) & \(\mbf{11}\) & \(\mbf{12}\) & \(\mbf{12}\) & \(\mbf{13}\) \\
        7 & 2 & 2 & 3 & 4 & 5 & 6 & 7 & 8 & \(\mbf8\) & \(\mbf9\) & \(\mbf{10}\) & \(\mbf{11}\) & \(\mbf{12}\) & \(\mbf{13}\) & \(\mbf{14}\) \\
        8 & 2 & 2 & 3 & 4 & 5 & 6 & 7 & 8 & 9 & \(\mbf9\) & \(\mbf{10}\) & \(\mbf{11}\) & \(\mbf{12}\) & \(\mbf{13}\) & \(\mbf{14}\) \\
        9 & 2 & 2 & 3 & 4 & 5 & 6 & 7 & 8 & 9 & 10 & \(\mbf{10}\) & \(\mbf{11}\) & \(\mbf{12}\) & \(\mbf{13}\) & \(\mbf{14}\)\\
        10& 2 & 2 & 3 & 4 & 5 & 6 & 7 & 8 & 9 & 10 & 11 & \(\mbf{11}\) & \(\mbf{12}\) & \(\mbf{13}\) & \(\mbf{14}\)\\
        11& 2 & 2 & 3 & 4 & 5 & 6 & 7 & 8 & 9 & 10 & 11 & 12 & \(\mbf{12}\) & \(\mbf{13}\) & \(\mbf{14}\)
    \end{tabular}\caption{Values of \protect$f(d,k)\protect$.  Numbers in bold are lower bounds.}
    \end{table}





    \begin{comment}
    \begin{figure}[hbt]
        \centering
            \includegraphics[width=.7\textwidth, page=14]{pictures.pdf}
        \caption{The polytope $\xp 3$ and a Gale transformation of its vertices.\label{Fig:xp3Gale}}
    \end{figure}
    \end{comment}
\begin{comment}
Note that it is possible to have an anticlique of size \(2\) for which (what the hell was I trying to say here?)

    If \(P\) is a \(d\)-polytope, and \(G\in\galed(P)\) is a Gale diagram of \(P\), then it is a simple matter to determine \(\gr P\), the graph of \(P\), from \(G\).  A \dfn{coedge} of a polytope \(P\) is a coface which corresponds to an edge of \(P\).  By Theorem \ref{Thm:CofaceIFF}, the set \(X=\seta{\ve x_1,\ve x_2}\sbset\vrt P\) (\(\ve x_1\ne\ve x_2\))  is a coedge if and only if \(\ol{ V\setminus X}\) either captures the origin, or is empty.  This condition can only fail if there is some hyperplane \(H\) containing \(\ve 0\) such that \(\ol X= G\cap H^{(+)}\).  Thus the nonedges of the graph of a polytope can be determined from a Gale diagram easily.
\end{comment}

    \chapter{Realizability of Complete Multipartite Graphs}
\label{chap:CompMulti}

The aim of this chapter is to answer the question, "When is a complete multipartite graph realizable?".  A complete answer will not be given, but progress will be made toward an answer.  In particular, a complete multipartite graph is the graph of a polytope if and only if if is either \(K_{1,1}\) or \(K_{2,2}\).  A characterization of all of the possible \(2\)- and \(3\)-faces will also be given.

\section{Hereditary Classes of Graphs}

    A class of graphs \(\mf G\) called a \dfn{hereditary} if \(G\in\mf G\), and \(H\) is an induced subgraph of \(G\) implies \(H\in\mf G\).

    \begin{Example}  Throughout these examples, \(0\notin\N\).
        \begin{enumerate}
            \item   Let \(q\in\N\).  The class of discrete graphs with at most \(q\) vertices is hereditary, and is denoted \(\mf D_q=\setb{D_n}{n\le q}\).
            \item   The class of all discrete graphs is hereditary, and is denoted \(\mf D=\setb{D_n}{n\in\N}\).
            \item   The class of all complete graphs is hereditary, and is denoted \(\mf K=\setb{K_n}{n\in\N}\).
            \item   The class of all complete bipartite graphs (along with the discrete graphs) is hereditary, and is denoted \(\mf K_2=\setb{K_{n,m}}{n,m\in\N}\cup\mf D\).
            \item   Let \(3\le q\in\N\). The class of all complete multipartite graphs with fewer than \(q\) parts (along with the discrete graphs) is hereditary, and is denoted
                    \[
                        \mf K_q=\setb{K_{n_1\dc n_q}}{n_i\in\N\text{ for each }i\in\brac q}\cup\mf K_{q-1}.
                    \]
            \item   Let \(a_1\le a_2\le\dotsb\le a_m\) be a sequence of natural numbers.  Then the class
                \[
                    \mf K[a_1,a_2\dc a_m]=\setb{K_{n_1,n_2\dc n_m}}{n_i\le a_i}\cup\mf K_{m-1}
                \]
                is hereditary.
            \item   Let \(a_1\le a_2\le\dotsb\le a_m\) be a sequence of natural numbers.  Then the class
                \[
                    \mf K(a_1,a_2\dc a_m)=\setb{K_{n_1,n_2\dc n_q}}{q\le m\text{ and }\exists i_1<i_2<\dotsb<i_q\le m\text{ with }n_{i_j}\le a_j}\cup\mf D_{a_m}
                \]
                is hereditary.  More over it is exactly the set of induced subgraphs of the complete multipartite graph \(K_{a_1,a_2\dc a_m}\).
        \end{enumerate}

        \begin{Theorem}\label{Thm:Heredity}
            Suppose \(\mf G\) is a hereditary class of graphs and there is some \(d\in\N\) such that no \(G\in\mf G\) is \(d\)-realizable.  Then for every \(d'>d\) there is no \(G\in\mf G\) which is \(d'\)-realizable.
        \end{Theorem}
        \begin{proof}
            Suppose \(\mf G\) is a hereditary class of graphs and there is some \(d\in\N\) such that no \(G\in\mf G\) is \(d\)-realizable.  Suppose further that there is some \(d'>d\) and a \(G\in\mf G\) such that \(G\) is \(d'\)-realizable.

            Let \(P\) be any \(d'\)-polytope whose graph is \(G\), and let \(F\) be any \(d\)-face of \(P\).  By Theorem \ref{Thm:Induced}, the graph of \(F\) is an induced subgraph of \(G\).  Since \(G\in\mf G\) and \(\mf G\) is hereditary, the graph of \(F\) must also be an element of \(\mf G\).  However this contradicts the assumption that no graph in \(\mf G\) is \(d\)-realizable.
        \end{proof}
    \end{Example}
\section{Complete Bipartite Graphs}

    This section will answer the question "For which values of \(n,m,d\) is the graph \(K_{n,m}\) \(d\)-realizable?".  First, recall the convention that a complete multipartite graph \(K_{n_1,n_2\dc n_t}\) is written such that \(n_1\le n_2\le\dotsb\le n_t\).  Then it will be shown that the answer to the above question is, "only if \(n=m=d\in\seta{1,2}\)".  These are the graphs of \(\simp1\) and \(\xp2\).  Recall that the connectivity of a graph \(\conn G\) is the largest \(q\) such that \(G\) is \(q\)-connected.

    \begin{Lemma}\label{Lem:ConnCompBip}
        \(\conn{K_{n,m}}=n\).
    \end{Lemma}
    \begin{proof}
        Write \(V(K_{n,m})=P\cup Q\) with \(P=\seta{p_1,p_2\dc p_n}\), \(Q=\seta{q_1,q_2\dc q_m}\) and \(E(K_{n,m})=\setb{p_iq_j}{p_i\in P\text{ and }q_j\in Q}\).  Then \(\deg(q_j)=n\) for every \(j\in\brac m\) and \(\deg(p_i)=m\) for every \(i\in\brac n\).  Thus the minimum degree of a vertex of \(K_{n,m}\) is \(n=\min\seta{n,m}\).  Hence \(\conn{K_{n,m}}\le n\).

        To show that \(\conn{K_{n,m}}\ge n\), a set of \(n\) disjoint paths will be constructed for each pair of vertices of \(K_{n,m}\).  There are three cases: a pair of vertices from \(P\); a pair of vertices from \(Q\); and a pair with one vertex in \(P\) and the other in \(Q\).

        In the case that both vertices of the pair are in \(P\), say \(p_i,p_j\), the paths
            \[
                p_i,q_t,p_j\qquad t\in\brac n
            \]
        are disjoint, and there are \(n\) of them.

        In the case that both vertices of the pair are in \(Q\), say \(q_i,q_j\), the paths
            \[
                q_i,p_t,q_j\qquad t\in\brac n
            \]
        are disjoint, and there are \(n\) of them.

        For the case that the pair of vertices is of the form \(p_i,q_j\) with \(p_i\in P\) and \(q_j\in Q\), assume, without loss of generality, that \(i=j=n\).  Then the paths
            \[
                p_n,q_t,p_t,q_n\qquad t\in\brac{n-1}
            \]
        are disjoint, and there are \(n-1\) of them.  These paths, together with the path \(p_n,q_n\) form \(n\) disjoint paths.
    \end{proof}

    \begin{Theorem}\label{Thm:CompBip}
        \(K_{n,m}\) is \(d\)-realizable if and only if \(n=m=d\in\seta{1,2}\).
    \end{Theorem}
    \begin{proof}
        Since \(\card{V(K_{n,m})}\ge 2\), \(K_{n,m}\) is not \((-1)\)-, or \(0\)-realizable.

        Suppose \(n=1\).  Then \(\conn{K_{1,m}}=1\), and thus \(K_{1,m}\) is realizable only if \(d=1\).  There is only one combinatorial type of \(1\)-polytope, and it has graph \(K_{1,1}\).

        Suppose \(n=2\).  Then \(\conn{K_{2,m}}=2\), and thus \(K_{2,m}\) is realizable only if \(d\le2\).  However, \(K_{2,n}\) is never \(1\)-realizable, so if \(K_{2,m}\) is to be \(d\)-realizable, then \(d=2\).  Each \(2\)-polytope has a graph which is a cycle, and thus each vertex has degree \(2\).  The only complete bipartite graph for which each vertex has degree \(2\) is \(K_{2,2}\), and it is the graph of \(\xp2\), the \(2\)-crosspolytope.

        Suppose \(n\ge3\).  Then \(K_{n,m}\) is  at least \(3\)-connected.  Suppose that \(K_{n,m}\) is \(3\)-realizable.  Steinitz's Theorem (Theorem \ref{Thm:Steinitz}) thus implies that \(K_{n,m}\) must be planar.  However this is not possible since \(K_{n,m}\) has an induced \(K_{3,3}\) for \(n\ge3\) (and therefore a \(K_{3,3}\) minor).  Thus \(K_{n,m}\) is never \(3\)-realizable.

        Ergo, by Theorem \ref{Thm:Heredity}, \(K_{n,m}\) is not \(d\)-realizable for \(d\ge3\).
    \end{proof}

\section{Complete \protect$3\protect$-partite Graphs}
    The proof of the following lemma is similar to that of lemma \ref{Lem:ConnCompBip}.
    \begin{Lemma}
        \(\conn{K_{1,n,m}}=n+1\)
    \end{Lemma}

    \begin{Theorem}\label{Thm:KOneNM}
        \(K_{1,n,m}\) is \(d\)-realizable if and only if \(K_{n,m}\) is \((d-1)\)-realizable.
    \end{Theorem}
    \begin{proof}
        Suppose that \(n=1\).  Then \(\conn{K_{1,1,m}}=2\), and \(K_{1,1,m}\) is only a cycle if \(m=1\).  In which case it is the graph of \(\simp2\), the \(2\)-simplex.

        Suppose that \(n=2\).  The graph \(K_{1,2,2}\) is the graph of \(\pyr{\xp2}\).  If \(m\ge 3\), then \(K_{1,2,m}\) cannot be the graph of a \(3\)-polytope since it would then be a \(3\)-polytope with \(3+m\) vertices and an \(m\)-anticlique (see Theorem \ref{Thm:Anticliques}).  The graph \(K_{1,2,m}\) also cannot be \(d\)-realizable for \(d>3\) since \(\conn{K_{1,2,m}}=3\).

        Suppose that \(n=3\).  The graph \(K_{1,3,m}\) then has an induced \(K_{3,3}\), and therefore cannot be planar (hence is not \(3\)-realizable).  Furthermore, \(K_{1,3,m}\) cannot be the graph of a \(4\)-polytope since it has \(1+3+m=4+m\) vertices and an \(m\)-anticlique.

        Suppose that \(n\ge4\), and write
            \[
                V(K_{1,n,m})
                    =
                        \seta{a}                \cup
                        \setb{b_i}{i\in\brac n} \cup
                        \setb{c_i}{i\in\brac m}.
            \]
        where each set in the union above is one of the sets of vertices in the definition of a complete bipartite graph.  Here again, \(K_{1,n,m}\) has an induced \(K_{3,3}\), and is therefore not \(3\)-realizable.  Suppose that \(K_{1,n,m}\) is the graph of a \(4\)-polytope \(P\).  Then the set of induced subgraphs of \(K_{1,n,m}\) is \(\mf K(1,n,m)\) which is hereditary.  The only graph in \(\mf K(1,n,m)\) that is \(3\)-realizable is \(K_{1,2,2}\), and therefore each facet of \(P\) must be a pyramid over a quadrilateral.  In order to obtain an induced \(K_{1,2,2}\) from \(K_{1,n,m}\), the set of vertices must be of the form \(\seta{a,b_{i_1},b_{i_2},c_{j_1},c_{j_2}}\).  However this forces the vertex \(a\) to be in each facet, and this is impossible.

        Thus no graph of the form \(K_{1,n,m}\) is \(4\)-realizable, and therefore no graph \(K_{1,n,m}\) is \(d\)-realizable for \(d\ge 4\).
    \end{proof}

    The graphs \(K_{2,2,m}\) are not \(4\)-realizable since they would then have \(2+2+m=4+m\) vertices and an \(m\)-anticlique.  More generally, \(K_{2,n,m}\) is not \((2+n)\)-realizable.

    The \(K_{2,n,m}\) case is more difficult.  First, note that a graph of this form is only planar if \(n=m=2\), and in this case is the graph of \(\xp3\), the \(3\)-crosspolytope.

    Suppose that \(K_{2,n,m}\) is the graph of some polytope \(P\).  Working with the hereditary class of graphs \(\mf K(2,n,m)\) (this is the set of induced subgraphs of \(K_{2,n,m}\)) shows that the \(2\)-faces of \(P\) must be either \(2\)-simplices, or \(2\)-crosspolytopes.  Similarly, the \(3\)-faces of \(P\) can only be pyramids over quadrilaterals, or \(3\)-crosspolytopes.  Further, \(P\) has at most one facet that is combinatorially equivalent to \(\xp3\) since each \(\xp3\) facet must contain both vertices in the color class of cardinality \(2\).  This restriction of the allowable facets does not show that there is no such polytope.  A proof of the following theorem can be found in \cite{PerShep}.
        \begin{Theorem}[Perles, Shephard]
            If \(P\) is a \(d\)-polytope with \(d+2\) or fewer vertices, then there is a \((d+1)\)-polytope all of whose facets are combinatorially equivalent to \(P\).
        \end{Theorem}
    The authors also prove, in the same paper, that for \(d\ge6\) there is no \((d+1)\)-polytope all of whose facets are combinatorially equivalent to \(\xp d\).

    Suppose \(P\) is a \(4\)-polytope with graph \(K_{2,3,3}\) and that \(P\) has a facet combinatorially equivalent to \(\xp3\).  Then by a careful consideration of the possible ridges, it can be shown that there must be a ridge that is contained in only one facet.  Therefore there is no \(4\)-polytope with graph \(K_{2,3,3}\), and hence no polytope with graph \(K_{2,3,3}\).  Alternatively, the two papers \cite{Alts:Quasi} and \cite{Alts:Complete} provide a complete list of all \(4\)-polytopes with \(8\) vertices.  These papers show that any such polytope is either quasisimplicial (all of its facets are simplicial), or has a facet that is a simplex.  Since a realization of \(K_{2,3,3}\) could satisfy neither of these properties, there can be no such polytope.

    \subsection{Complete \protect$k\protect$-partite Graphs \protect$(k>3)\protect$}
    In general, if \(P\) is a realization of a complete multipartite graph, then every one of its \(2\)-faces must be combinatorially equivalent to either \(\simp2\) or \(\xp2\) since these are the only complete multipartite graphs that are cycles.  Similarly, every one of its \(3\)-faces must be combinatorially equivalent to one of the following:
        \begin{itemize}
            \item   a \(3\)-simplex, or
            \item   a pyramid over \(\xp2\), or
            \item   a bipyramid over a \(2\)-simplex, or
            \item   a \(3\)-crosspolytope
        \end{itemize}
    since these are the only planar, \(3\)-connected complete multipartite graphs.

    A direct consequence of Theorem \ref{Thm:Anticliques} is that if \(K_{n_1,n_2\dc n_t}\) is \(d\)-realizable, then \(d<\sum_{i\in\brac{t-1}}n_i\).  If \(K_{n_1,n_2\dc n_t}\) is \(d\)-realizable as a polytope \(P\), then \(K_{1,n_1,n_2\dc n_t}\) is \(d+1\)-realizable as a pyramid over \(P\).  The converse is, unfortunately, not true since \(K_{1,1,1,2}\) is the graph of \(\cyc53\) (i.{}e.{}, a bipyramid over a \(2\)-simplex) but \(K_{1,1,2}\) is not the graph of any polytope.  Similarly, \(K_{n_1,n_2\dc n_t,2}\) is realizable as a bipyramid over \(P\).

    Thus, if (as a set) \(\seta{n_1,n_2\dc n_t}\sbset\seta{1,2}\), then \(K_{n_1,n_2\dc n_t}\) is the graph of a polytope if and only if (as a multiset) \(\seta{n_1,n_2\dc n_t}\) is not equal to either \(\seta{1,2}\) or \(\seta{1,1,2}\).

    \begin{Conjecture}
        If \(K_{n_1,n_2\dc n_t}\) is the graph of a polytope, then \(\seta{n_1,n_2\dc n_t}\sbset\seta{1,2}\) as sets,.
    \end{Conjecture}

    Note that the conjecture does not make mention of the dimension of the polytope.  This is because the graphs of higher-dimensional crosspolytopes can be realized in multiple dimensions.

\section{Graphs of Crosspolytopes}
    Throughout this section, let \(G_n\) denote the graph of the \(n\)-dimensional crosspolytope for \(n\ge 2\), that is
        \[
            G_n
                =   \gr{X_n}
                =   K_{\underbrace{2,2\dc2}_n}.
        \]
    The goal of this section will be to establish for which \(d\) the graph \(G_n\) is \(d\)-realizable.

    The graphs \(G_2\) and \(G_3\) are realizable only in dimensions \(2\) and \(3\) respectively.  Thus assume throughout that \(n\ge 4\).  The first thing to do is establish an upper bound on the values \(d\) such that \(G_n\) is \(d\)-realizable.  Recall the Hadwiger number of a graph \(G\) is \(\had G=\max\setb{n}{K_n\text{ is a minor of }G}\).  Theorem \ref{Thm:CompleteMinor} then implies:
    \begin{Cor}\label{Cor:HadBound}
        If a graph \(G\) is \(d\)-realizable, then \(d\le\had G-1\).
    \end{Cor}

    Thus a natural question to ask is "What is \(\had{G_n}\)?".  This question was answered in \cite{Halin}.

    \begin{Theorem}[Halin 1966]\label{Thm:Halin}
        \[\had{G_n}=\floor{\frac{3n}2}.\]
    \end{Theorem}
        The following gives, for each \(n\), an explicit construction of a maximal complete minor of \(G_n\).

        Write \(V(G_n)=V_1\cup V_2\) where \(V_1=\seta{v^1_1,v^1_2\dc v^1_n}\) and \(V_2=\seta{v^2_1,v^2_2\dc v^2_n}\) are disjoint sets such that
            \begin{align*}
                E(G_n)
                    &=  \setb{v^i_jv^k_l}{i,k\in\brac2,\,j,l\in\brac n\text{, and if }i\ne k\text{, then }j\ne l}\\
                    &=  \binom{V(G_n)}2\setminus\setb{v^1_jv^2_j}{j\in\brac n}.
            \end{align*}
        Set
            \begin{align*}
                D
                    &=  \setb{v^i_jv^2_k}{i\in\brac2,\,j,k>\floor{n/2}}
                            \cup    \setb{v^i_jv^2_k}{i\in\brac2,\,j\le\floor{n/2}\text{ and }k>\floor{n/2}\text{ with }j+k\ne n+1}
            \end{align*}
        and
            \begin{align*}
                C
                    &=  \setb{v^2_iv^2_j}{i+j=n+1}.
            \end{align*}
        Then \((G_n\setminus D)/C\) is a complete graph with \(\floor{3n/2}\) vertices.  The paths
            \begin{align*}
                &   v^1_iv^1_j              &   &   i,j\in\brac n\\
                &   v^2_iv^2_j              &   &   i,j\le\floor{n/2}\\
                &   v^1_iv^2_j              &   &   i\in\brac n, j\le\floor{n/2}\text{ with }i\ne j\\
                &   v^1_iv^2_{n+1-i}v^2_i   &   &   i\le\floor{n/2}
            \end{align*}
        contract to form the edges in the complete graph.  Thus \(h(G_n)\ge\floor{3n/2}\).  This construction is included here because \cite{Halin} does not give an explicit construction of a maximal complete minor.  See Figures \ref{Fig:Xfour} and \ref{Fig:Xfive} for examples of these constructions.
        \begin{figure}[h!bt]
            \centering
                \includegraphics[width=.5\textwidth, page=30]{pictures.pdf}
            \caption{\protect$G_4\protect$, edges in the deletion set are dotted, and those in the contraction set are dashed.\label{Fig:Xfour}}
        \end{figure}

        \begin{figure}[h!bt]
            \centering
                \includegraphics[width=.7\textwidth, page=31]{pictures.pdf}
            \caption{\protect$G_5\protect$, edges in the deletion set are dotted, and those in the contraction set are dashed.\label{Fig:Xfive}}
        \end{figure}
        \subsection{High Dimensional Realizations}

            The graph \(G_n\) \emph{could} thus be realizable in each dimension from \(4\) to \(\floor{3n/2}-1\).  Before giving explicit constructions of \(d\)-polytopes with graph \(G_n\) for \(n\le d<\floor{3n/2}\), notice that if \(p_i\in\N\) with \(p_i\ge 2\) and \(\sum_{i\in\brac m}p_i=q\), then
                \[
                    \gr{\xp{p_1}\join\xp{p_2}\join\dotsb\join\xp{p_m}}=G_q
                \]
            and
                \[
                    \dim(\xp{p_1}\join\xp{p_2}\join\dotsb\join\xp{p_m})=\sum_{i\in\brac m}p_i+m-1.
                \]
            In particular, let \(\xp2^k\) be the join of \(k\) copies of \(\xp2\), so that \(\gr{\xp2^k}=G_{2k}\) and \(\dim\xp2^k=3k-1\).  Thus the polytope \(\xp{3n-2d}\join\xp2^{d-n}\) is a \(d\)-polytope with graph \(G_n\).  These polytopes are by no means unique with these properties.  Others can be constructed by similar means by taking joins of crosspolytopes (or even polytopes with these graphs) and forcing the joins to have the required number of vertices and be of the required dimension.  That is:
                \begin{Theorem}
                    The graph \(G_n\) is \(d\)-realizable for \(n\le d<\ds\floor{\frac{3n}2}\).
                \end{Theorem}
            
            This settles the question of \(d\)-realizability of \(G_n\) for \(d\ge n\).  However, the number of combinatorial types of \(d\)-polytopes with graph \(G_n\) and \(d\ge n\) is not known.  These constructions merely provide a lower bound for that number.


            The following is a Corollary to Theorems \ref{Thm:Anticliques} and \ref{Thm:Halin}.
                \begin{Cor}
                    Let \(P\) be a simple \(d\)-polytope that is not a simplex. If \(\gr P=K_{n_1,n_2\dc n_s}\), then \(P=X_2\).
                \end{Cor}
                \begin{proof}
                    If \(P\) is simple, then \(\deg\ve v=d\) for every \(\ve v\in\vrt P\).  Therefore \(\sum_{j\in\brac s\setminus\seta i}n_j=d\) for every \(i\in\brac s\).  Hence \(n_1=n_2=\dotsb=n_s=k\).  Since \(P\) is not a simplex, \(k\ne1\).  Notice that \(P\) is  a \(d\)-polytope with \(d+k\) vertices whose graph has a \(k\)-anticlique.  Thus \(k=2\).  Now, \(P\) is a \(2(s-1)\)-polytope with graph \(G_s\).  Therefore
                        \[
                            2(s-1)\le\floor{\frac{3s}2}-1\le\frac{3s}2-1.
                        \]
                    Solving for \(s\) yields \(s\le2\).  Hence \(s=2\), whence \(\gr P=G_2\).  Thence \(P=X_2\).
                \end{proof}

        \subsection{Low Dimensional Realizations}

            In \cite{GrunBook}, a construction of a \(4\)-polytope \(Z\) with graph \(G_5\) is given.  Repeated bipyramids over this polytope yield \((n-1)\)-realizations of \(G_n\) for \(n\ge 5\).  Note that \(Z\join \xp2\) is a \(7\)-polytope with graph \(G_7\) and is not \(\xp7\).  Thus, in general, the graph \(G_n\) is not the graph of a unique \(n\)-polytope.  The polytope \(Z\) has an additional property that will be useful in talking about low dimensional realizations of \(G_n\).

            \begin{Definition}
                A polytope \(P\) is called \dfn{centrally symmetric} if there is a point \(\ve p\in P\) such that \(\ve x\in P-\ve p\) implies \(-\ve x\in P-\ve p\).
            \end{Definition}

            It will be assumed throughout that \(\ve p=\ve 0\).  The \(4\)-realization \(Z\) of \(G_5\) is centrally symmetric and much of the work on low-dimensional realizations of \(G_n\) concerns centrally symmetric realizations.   In \cite{GrunBook} it is shown that if \(G_6\) is \(4\)-realizable by a polytope \(P\), then \(P\) cannot be centrally symmetric.

            Let \(P\) be a centrally symmetric \(d\)-polytope with graph \(G_n\) with \(n<d\).  The polytope \(P\join\xp2\) is a \((d+3)\)-polytope with graph \(G_{n+2}\).  However, \(P\join\xp2\) is not centrally symmetric.

            It is shown in \cite{BarNov:CSCyclic} that if \(P\) is a  centrally symmetric \(d\)-polytope with \(N\) vertices, then
                \[
                    f_1(P)
                        \le
                            \frac{N^2}2(1-2^{-d}).
                \]
            In the case that \(\gr P=G_n\), this implies that \(\binlog n\le d\) where \(\binlog n\) is the binary logarithm \(\log_2 n\).

            Let \(m\in\N\).  The paper \cite{BarLeeNov:Many} gives an explicit construction of a \(d\)-dimensional centrally symmetric polytope with graph \(G_n\) where
                \begin{align*}
                    d
                        &=
                            4m+6\\
                    n
                        &=
                            2\cdot3^{m+1}.
                \end{align*}
            This construction, in the case \(m=0\), yields that the polytope \(\xp6\) has graph \(G_6\).  Setting \(m=1\) gives a centrally symmetric \(10\)-polytope with graph \(G_{18}\).  By considering bipyramids over these polytopes, one obtains centrally symmetric polytopes of dimension \(4m+7\) with graph \(G_n\) where \(n=2\cdot3^{m+1}+1\).  Furthermore, taking a join  with \(\xp2\) yields a polytope of dimension \(4m+11\) and with graph \(G_n\) where \(n=2\cdot3^{m+1}+2\). 
    \include{Chapter_Conclusion/Ch_Conc}
    \nocite{StanleyEC1}
    \nocite{StanleyEC2}
    \nocite{Handbook}
    \nocite{West}
    \nocite{BarNov:Nbly}
    \nocite{BarNov:Mom}
    \nocite{Kalai:Simple}
    \nocite{Hibi}
    \nocite{BayerLee}
    \nocite{Ewald}

    \global\long\def\bibname{References}


    \bibliographystyle{apalike2}
    \bibliography{Biblio/allcites}


    \appendix
    
\chapter{Python Code to Determine Whether or not a Point Configuration is a Gale Diagram}\label{Appendix:Program}
This program was written in Python version \(2.7.2\).
\lstset{language=Python, breaklines=true}
\lstinputlisting{Gale.py}

\end{document}
